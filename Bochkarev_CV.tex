\documentclass[11pt]{article} \usepackage{geometry} % margins
\usepackage{titlesec} % styling sections
\usepackage[fixed]{fontawesome5} % nice icons
\usepackage[svgnames]{xcolor}
\usepackage{hyperref}  % for hyperlinks

% styling parameters >>> Configure page margins with geometry
\geometry{left=1.4cm, top=1cm, right=1.4cm, bottom=1.2cm, footskip=0.5cm}
\hypersetup{
  colorlinks=true,
  allcolors={NavyBlue}
}

\titleformat{\section}{\normalfont\Large\bfseries}{\thesection}{1em}{}[{\titlerule[0.3pt]}]
\titlespacing*{\section}{0pt}{0pt}{10pt}
\pagenumbering{gobble}
% <<<

% some formatting macros
\newcommand{\edu}[3]{
  \textbf{#1}\hfill (#2)\\
  #3\vspace{0.7em}}

\newcommand{\jobl}[5]{%
  \textbf{#1}\hfill (#2)\\
  #3\vspace{0.25em}\\
  \textbf{Role:} #4\\
  \textbf{Focus:} #5}

\newcommand{\job}[5]{%
  \jobl{#1}{#2}{#3}{#4}{#5}\vspace{1.5em}
}

\newcommand{\mhref}[1]{\hfill\href{#1}{\small (more\faExternalLink*)}}
\newcommand{\lastupdate}{\vfill\hfill \tiny Updated: 2021-09-27\newline \hspace*{\fill} Alexey Bochkarev}
\begin{document}
% info header >>>
  \begin{minipage}[c]{0.74\textwidth}
    \centering
    {\LARGE \textbf{Alexey Bochkarev}}\\
    ---\\ 
    Researcher in Mathematical Optimization / Operations Research\\
    PhD candidate, Clemson University
  \end{minipage}\hfill%
  \begin{minipage}{0.26\textwidth}
    \faEnvelope \href{mailto:abochka@g.clemson.edu}{abochka@g.clemson.edu}\\
    \faGlobe \href{https://www.bochkarev.io}{www.bochkarev.io}\\
    \faGithub \href{https://github.com/alex-bochkarev}{alex-bochkarev}\\
    \faTwitter \href{https://twitter.com/a_bochka}{@a\_bochka}\\
    \faTelegram \href{https://t.me/abochka}{@abochka}
  \end{minipage}
% <<< end info header
  
  \vspace{1.0em}
  \section*{Research interests}
  Mathematical optimization, theory and applications, especially:\vspace{0.3em}
  \begin{itemize}
    \begin{minipage}{0.5\linewidth}
      \item Combinatorial optimization,
      \item Network optimization and interdiction,
    \end{minipage}
    \begin{minipage}{0.5\linewidth}
      \item Decision diagrams and dynamic programming,
      \item Applications of reinforcement learning techniques.
    \end{minipage}
  \end{itemize}

  \noindent
  \textbf{Applications:} So far my research has been driven more by
  methodological questions, but I do have some experience of
  implementing applied models in industry. Also, due to my background I have a
  special interest in optimization related to electricity markets:
  pricing / OPF / economic dispatch / infrastructure planning, etc.\vspace{0.5em}
  \section*{Research experience / current projects \mhref{https://www.bochkarev.io/research/}}
  \begin{itemize}
    \itemsep0pt
  \item \textbf{Align-BDD:} seeking to obtain computational benefits and
    sensitivity information by representing a combinatorial problem as a
    collection of Binary Decision Diagrams (BDDs). The project involves creating a heuristic to enforce a
    certain structural property for a pair of BDDs and building a related
    ``computational pipeline'' for a specific, hard optimization problem: a
    variant of the facility location. 
  \item \textbf{DSPI:} applying game-playing and reinforcement
    learning techniques to the Dynamic Shortest-path Interdiction problem,
    in a framework of a Monte-Carlo Search Tree based algorithm.
  \end{itemize} 
  Both projects involve design and implementation of an algorithm and
  the related computational experiments.\vspace{0.5em}

  \noindent\textbf{Research supervisor:}
  \href{https://scholar.google.com/citations?user=87CaUHYAAAAJ&hl=en}{Dr. J. Cole
    Smith.}
  \vspace{0.5em}

 \section*{Working papers and conference presentations}
 \begin{itemize}
    \itemsep0pt
    \item \textbf{Working paper:} A. A. Bochkarev, J.C. Smith, On Aligning
    Non-Order-Associated Binary Decision Diagrams, revision submitted to
    \textit{INFORMS Journal on Computing}. (First submission: 2020/08/22)
    \item \textbf{Presentation:} INFORMS Annual Meeting, 2020 (virtual), BDD section.
 \end{itemize}
 \section*{Grants and awards}
 \begin{itemize}
 \itemsep0pt 
    \item Clemson University Doctoral Disseration Completion Grant (support for Fall 2021)
    \item Graduate Travel Grant (to participate in INFORMS Annual Meeting 2021)
    \item International Teaching Fellowship from Clemson University (partial
      support in 2020, training in teaching)
 \end{itemize}
 \noindent
 \begin{minipage}[t]{0.49\textwidth}
   \section*{Education} 
   \edu{PhD Industrial Engineering}{2018--2021 exp}{
     Clemson University, US\\
     Operations Research track}

   \edu{MSc/BSc Appl. Math and Physics}{2004--2010}{
     Moscow Institute of Physics\\
     and Technology, Russia}

   \edu{M.A. Economics}{2008--2010}{
   New Economic School, Russia}
 % Selected coursework included:

 % \textbf{University:} Mathematical Programming, Network Flows, Algorithms and Data Structures,
 % Stochastic Programming intro, Power Markets and Regulations, Game Theory,
 % Probability / Statistics, Foundations of Data Science, Calculus \& Co.\vspace{0.5em}

 % \textbf{Online:} Analytics Edge (MITx @ EdX), ML basics (Andrew Ng @ Coursera).\\
\section*{(Human) Languages}
English (fluent), Russian (native), German (A1).

\end{minipage}\hfill%
\begin{minipage}[t]{0.49\textwidth}
   \section*{Technical skills \mhref{https://www.bochkarev.io/notes/stack/}}
   \textbf{Main programming stack:}
   \begin{itemize}
     \itemsep0pt
   \item Python (gurobi, CBC, numpy/pandas, etc.)
     \item R (ggplot, dplyr, tidyverse),
     \item Julia (JuMP/gurobi, LightGraphs),
     \item C++ (gurobi, armadillo/BLAS, boost).
   \end{itemize}\vspace{0.5em}
   \textbf{Basic knowledge:} PyTorch, Java, Matlab/Octave. \vspace{1em}

   \textbf{Other technical skills:}
   PBS (comp cluster), GNU/Linux, bash; make, git, \LaTeX, Emacs, basic GIS
   (QGIS), Inkscape, beamer / PPT / reveal.js, Jupyter.
   \end{minipage}

   \section*{Teaching experience \mhref{https://www.bochkarev.io/teaching/}}
   \begin{itemize}
     \itemsep0pt
      \item Designed and delivered two 4-days mini-courses aimed at gifted
      high-school students and early undergrads for School for Molecular and
      Theoretical Biology (SMTB) and Puschino Winter School (ZPSh), both Russian
      and English track:
      \begin{itemize}
        \itemsep0pt
        \item ``Practical Introduction to Probability Theory'', ZPSh-2021,
          SMTB-2021
        \item ``A Glimpse into Algorithms'', SMTB-2020 (workshop), SMTB-2021 (course)
      \end{itemize}
      \item TA in ``Intro probability'' undergrad course at Clemson University
        (IE3600), Summer 2021 \vspace{0.3em}
\end{itemize}
   \section*{Service and volunteering / Community}
   Besides teaching at summer and winter schools (above), I have been doing some work under the umbrella of Clemson University INFORMS Student Chapter:
   \begin{itemize}
     \itemsep0pt
   \item serving on the Executive Board: as a Secretary (2020) and President
     (2021),
   \item organized a ``Journal club on Network optimization and interdiction''
     (2021),
   \item designed and delivered ``OR Tech Seminar'' -- a series of four
     workshops on ``research toolbox'' (2021). 
   \end{itemize}
   \vspace{0.3em}

   \section*{Industry experience}
   \job{The Federal Grid (FGC UES)}{2013--2017}{Electricity transmission,
     Moscow, Russia}{Team head, modeling and analytics}{Performance
     benchmarking (branches), operational efficiency improvement. Internal
     regulations/KPI, strategy, analytics / modeling, and presentations.}

   \noindent
   \job{Ministry of Energy of Russia}{2013}{Public service: energy
     (electricity), Moscow, Russia}{Deputy team head}{Electricity transmission and
     distribution grids, benchmarking, economic efficiency.
     Analytics, presentations.}

   \noindent
   \job{Roland Berger Strategy Consultants GmbH}{2010--2013}{Strategic consulting,
     Moscow, Russia}{ Intern $\rightarrow$ Junior
     Consultant $\rightarrow$ Consultant}{Infrastructure and construction.
     Strategy and performance: market entry, supply/demand modeling, growth
     strategy, efficiency improvement. Internal knowledge sharing, modeling, presentations.}

   \lastupdate
\end{document}