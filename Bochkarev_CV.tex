\documentclass[11pt, a4paper]{article} \usepackage{geometry} % margins + paper size
\usepackage{titlesec} % styling sections
\usepackage[fixed]{fontawesome5} % nice icons
\usepackage[svgnames]{xcolor} % color names
\usepackage{hyperref}  % for hyperlinks

% styling parameters >>> Configure page margins with geometry
\geometry{left=1.4cm, top=1cm, right=1.4cm, bottom=1.2cm, footskip=0.5cm}
\hypersetup{
  colorlinks=true,
  allcolors={NavyBlue}
}

\titleformat{\section}{\normalfont\Large\bfseries}{\thesection}{1em}{}[{\titlerule[0.3pt]}]
\titlespacing*{\section}{0pt}{0pt}{10pt}
\pagenumbering{gobble}
% <<<

% some formatting macros
\newcommand{\edu}[3]{
  \textbf{#1}\hfill (#2)\\
  #3\vspace{0.7em}}

\newcommand{\jobl}[5]{%
  \textbf{#1}\hfill (#2)\\
  #3\vspace{0.25em}\\
  \textbf{Role:} #4\\
  \textbf{Focus:} #5}

\newcommand{\job}[5]{%
  \jobl{#1}{#2}{#3}{#4}{#5}\vspace{0.5em}
}

% fancy hyperlinks to external resources.
\newcommand{\mhref}[1]{\hfill\href{#1}{\small (more\faExternalLink*)}}

\begin{document}
% info header >>>
  \begin{minipage}[c]{0.74\textwidth}
    \centering
    {\LARGE \textbf{Alexey Bochkarev}}\\
    ---\\
    Researcher in Mathematical Optimization / Operations Research\\
    Postdoc at RPTU Kaiserslautern :: \href{https://math.rptu.de/en/wgs/opt/research}{AG Optimierung} (Germany)
  \end{minipage}\hfill%
  \begin{minipage}{0.26\textwidth}
    \faEnvelope \href{mailto:a@bochkarev.io}{a@bochkarev.io}\\
    \faGlobe \href{https://www.bochkarev.io}{www.bochkarev.io}\\
    \faGithub \href{https://github.com/alex-bochkarev}{alex-bochkarev}\\
    \faTwitter \href{https://twitter.com/a_bochka}{@a\_bochka}\\
    \faTelegram \href{https://t.me/abochka}{@abochka}
  \end{minipage}
% <<< end info header

  \vspace{1.0em}
  \section*{Research interests}
  \textbf{Mathematical optimization}, theory and applications, especially:\vspace{0.3em}
  \begin{itemize}
    \begin{minipage}{0.5\linewidth}
      \item Combinatorial optimization,
      \item Network optimization and interdiction,
    \end{minipage}
    \begin{minipage}{0.5\linewidth}
      \item Decision diagrams and dynamic programming,
      \item Applications of reinforcement learning techniques.
    \end{minipage}
  \end{itemize}


  \noindent \textbf{Quantum computing}, its applications and efficiency for optimization.\vspace{1em}

  \noindent
  \textbf{Applications,} I have a special interest in optimization related to
  electricity markets: pricing / OPF / economic dispatch / planning,
  etc.\vspace{1em}

 \noindent
 \begin{minipage}[t]{0.48\textwidth}
   \section*{Education}
   \edu{PhD Industrial Engineering}{2018--2021}{
     Clemson University, US\\
     Operations Research track}

  \noindent\textbf{Dissertation:} ``Selected Topics in Network Optimization:
  Aligning Binary Decision Diagrams for a Facility Location Problem and a Search
  Method for Dynamic Shortest Path Interdiction.''\\
  \noindent (\href{https://tigerprints.clemson.edu/all_dissertations/2915}{\footnotesize https://tigerprints.clemson.edu/all\_dissertations/2915})
  \vspace{0.5em}

\textbf{Research supervisor:}
  \href{https://scholar.google.com/citations?user=87CaUHYAAAAJ&hl=en}{Dr. J. Cole
    Smith.}\vspace{0.7em}

   \edu{MSc Appl. Math and Physics}{2004--2010}{
     Moscow Institute of Physics\\
     and Technology, Russia}

   \edu{M.A. Economics}{2008--2010}{
   New Economic School, Russia}

\end{minipage}\hfill%
\begin{minipage}[t]{0.49\textwidth}
   \section*{Technical skills \mhref{https://www.bochkarev.io/notes/stack/}}
   \textbf{Main programming stack:}
   \begin{itemize}
     \itemsep0pt
   \item Python (gurobi, CBC, numpy/pandas, etc.)
     \item R (ggplot, dplyr, tidyverse),
     \item Julia (JuMP/gurobi, LightGraphs),
     \item C++ (gurobi, armadillo/BLAS, boost).
   \end{itemize}\vspace{0.5em}
   \textbf{Basics:} PyTorch, Java, Matlab/Octave. \vspace{0.5em}

   \textbf{Other technical skills:}
   PBS (comp cluster), GNU/Linux, bash; make, git, \LaTeX, Emacs, basic GIS
   (QGIS), Inkscape, beamer / PPT / reveal.js, Jupyter.\vspace{0.7em}

    \section*{(Human) Languages}
    English (fluent), Russian (native), German (A1--A2).

   \end{minipage}
   \vspace{1em}

  \section*{Research experience and current projects \mhref{https://www.bochkarev.io/research/}}
  \begin{itemize}
    \itemsep0pt
    \item \textbf{Dynamic Shortest-Path Interdiction (DSPI):} (ongoing) applying
          game-playing and reinforcement learning techniques to DSPI problem, in
          a framework of a Monte-Carlo Search Tree based algorithm.
          \\(with
          \href{https://scholar.google.com/citations?user=87CaUHYAAAAJ&hl=en}{Dr.
          J. Cole Smith.})
    \item \textbf{Quantum Computing for Discrete Optimization:} (ongoing)
          highlighting three specific technologies (QAOA, Quantum Annealing, and
          Rydberg-blockade based regime) and applying them to a few discrete
          optimization problems (TSP, Max Cut, and Max Independent Set). I try
          to take an OR scientist perspective and discuss the possible
          workflows, issues, and prospects.
          \\(with
          \href{https://scholar.google.com/citations?user=yYeHbcIAAAAJ&hl=en}{Dr.
          Anita Schoebel} et al.)
    \item \textbf{Align-BDD:} seeking to obtain computational benefits and
          sensitivity information by representing a combinatorial problem as a
          collection of Binary Decision Diagrams (BDDs). The project involves
          creating a heuristic to enforce a certain structural property for a
          pair of BDDs and building a related ``computational pipeline'' for a
          specific, hard optimization problem: a variant of the facility
          location.
          \\(with
          \href{https://scholar.google.com/citations?user=87CaUHYAAAAJ&hl=en}{Dr.
          J. Cole Smith.})
  \end{itemize} \vspace{0.5em}

  \noindent\textbf{Current supervisor:} \href{https://scholar.google.com/citations?user=yYeHbcIAAAAJ&hl=en}{Dr.
          Anita Schoebel}
  \vspace{1.5em}

 \section*{Papers}
 \begin{itemize}
    \itemsep0pt
    \item \underline{A. A. Bochkarev}, J.C. Smith, (2023) On Aligning
    Non-Order-Associated Binary Decision Diagrams, accepted to \textit{INFORMS Journal on Computing}, online: \href{https://doi.org/10.1287/ijoc.2023.1293}{https://doi.org/10.1287/ijoc.2023.1293}.
  \item \underline{A. A. Bochkarev}, J.C. Smith, A Monte Carlo Tree Search for
    Dynamic Shortest-Path Interdiction, submitted to \textit{Networks} (under review).
  \item \underline{A. A. Bochkarev}, R. Heese, S. Jaeger, P. Schiewe, A. Schoebel, Quantum approaches for discrete optimization: a highlight of three technologies (in preparation).
 \end{itemize}
 \section*{Presentations / Talks}
 \begin{itemize}
   \itemsep0pt
   \item A case-based comparison of three key quantum approaches to discrete
         optimization, \textit{OR 2023} (Annual conference of GOR), Hamburg, Germany.
 \item A Monte Carlo Tree Search for Dynamic Shortest-Path Interdiction,
   \textit{International Network Optimization Conference, 2022}, Aachen, Germany
   (\href{https://sites.google.com/view/inoc2022/schedule}{INOC-2022}).
 \item On Aligning Non-Order-Associated Binary Decision Diagrams, \textit{INFORMS Annual
   Meeting, 2020} (virtual), BDD section.
 \end{itemize}

 \section*{Grants and awards}
 \begin{itemize}
 \itemsep0pt 
    \item Clemson University Doctoral Disseration Completion grant (support for Fall 2021)
    \item The Seth Bonder Foundation grant (to participate in INFORMS Annual Meeting 2021)
    \item International Teaching Fellowship from Clemson University (partial
      support in 2020, teaching training)
 \end{itemize}

   \section*{Teaching experience \mhref{https://www.bochkarev.io/teaching/}}
   \begin{itemize}
     \itemsep0pt
     \item Currently designing two 5 CP (125 hours) courses for distance
           learning MSc study program: ``Mathematical Foundations of Quantum
           Technologies'' and ``Quantum Computing I.''
      \item Designed and delivered three 4-days mini-courses/workshops aimed at gifted
      high-school students and early undergrads for School for Molecular and
      Theoretical Biology (SMTB) and Puschino Winter School (ZPSh), mostly in English
      (sometimes in Russian as well):
      \begin{itemize}
        \itemsep0pt
        \item ``Practical Introduction to Probability Theory,'' ZPSh-2021,
          SMTB-2021
        \item ``A Glimpse into Algorithms,'' SMTB-2020; SMTB-2021, SMTB-2022
        \item ``How to teach machines: simple examples on ML,'' SMTB-2022
      \end{itemize}
      \item TA in ``Intro probability'' undergrad course at Clemson University
        (IE3600), Summer 2021
\end{itemize}
   \section*{Service and volunteering / Community}
   Besides teaching at summer and winter schools (above), I have been doing some work under the umbrella of Clemson University INFORMS Student Chapter:
   \begin{itemize}
     \itemsep0pt
   \item serving on the Executive Board: as a Secretary (2020) and President
     (2021),
   \item organized a ``Journal club on Network optimization and interdiction''
     (2021),
   \item designed and delivered ``OR Tech Seminar'' -- a series of four
     workshops on ``research toolbox'' (2021).
   \end{itemize}
   \vspace{0.3em}

   \section*{Industry experience}
   \job{Electric energy / The Federal Grid (FGC UES)}{2013--2017}{Electricity transmission.
     Moscow, Russia}{Team deputy head $\rightarrow$ head; modeling and analytics}{Performance
     benchmarking (branches), operational efficiency improvement. Internal
     and external regulations / KPI, strategy, analytics / modeling, and presentations.}

   \noindent
   \job{Roland Berger Strategy Consultants GmbH}{2010--2013}{Strategic consulting.
     Moscow, Russia}{ Intern $\rightarrow$ Junior
     Consultant $\rightarrow$ Consultant}{Infrastructure and construction.
     Strategy and performance: market entry, supply/demand modeling, growth
     strategy, efficiency improvement. Internal knowledge sharing, modeling, presentations.}
   {%%% footer
   \vfill\noindent\tiny\LaTeX{} source: \href{https://github.com/alex-bochkarev/AB-CV}{Github}\hfill  Updated: 2023-09-15}
\end{document}
